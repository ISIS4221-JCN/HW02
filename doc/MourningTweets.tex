\section{Mourning Tweets}

El desarrollo y código fuente de esta sección de la tarea se encuentra en el cuaderno \texttt{HW02\_4.ipynb} adjunto.

\subsection{Mourning Lexicon}

\subsubsection{Processing Tweets}

Antes de realizar la construcción de los \textit{Lexicones} e implementar los modelos de clasificación se realizó un proceso estándar para \textit{tokenizar} los \textit{tweets}. Para esto se hizo un procesamiento especial pensado en lenguaje especifico que se utiliza en \textit{Twitter} y las particularidades de ruido que este medio conlleva. Para esto se realizaron los siguientes pasos:

\begin{itemize}
    \item Remover la puntuación de los \textit{Tweets}.
    
    \item \textit{Tokenizar} el \textit{dataset} con el \textit{tokenizador} (\texttt{TweetTokenizer}) especializado de \texttt{nltk} para este medio. Este:
    \begin{itemize}
        \item Estandariza todas las palabras a minúsculas.
        
        \item Reduce la longitud de las palabras en caso de repetir letras.
        
        \item Remueven caracteres especiales de \textit{Twitter} y usuarios (\textit{@}).
    \end{itemize}
    
    \item Se remueven \textit{stop\_words} utilizando el set de palabras de \texttt{nltk} para cada uno de los idiomas (inglés y español).
    
    \item Con los términos (\textit{tokens}) de los \textit{tweets} se construye un vocabulario (o \textit{dictionary}) para cada uno de los \textit{datasets} (inglés y español). 
    
    \item Con la función \texttt{filter\_extremes} de \texttt{Gensim} se remueven los términos (\textit{tokens}) menos comunes (aquellos que aparecen en menos de 3 documentos/\textit{tweets}).
    
\end{itemize}

\subsubsection{Construcción de lexicones}

Una vez se tienen definido el diccionario para los dos datasets, se construyen los lexicones estimando la probabilidad de que cada uno de los términos (\textit{tokens}) este en cada clase (luto o no luto). Para esto se utiliza el la verosimilitud  (\textit{likelihood}) de esta palabra: 

\begin{equation}
    P(w|c) = \frac{f(w,c)}{f(w,c)}
\end{equation}

\subsubsection{Resultados}

\subsection{Clasificadores}


\subsection{Análisis de importancia de características}